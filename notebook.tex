
% Default to the notebook output style

    


% Inherit from the specified cell style.




    
\documentclass[11pt]{article}

    
    
    \usepackage[T1]{fontenc}
    % Nicer default font (+ math font) than Computer Modern for most use cases
    \usepackage{mathpazo}

    % Basic figure setup, for now with no caption control since it's done
    % automatically by Pandoc (which extracts ![](path) syntax from Markdown).
    \usepackage{graphicx}
    % We will generate all images so they have a width \maxwidth. This means
    % that they will get their normal width if they fit onto the page, but
    % are scaled down if they would overflow the margins.
    \makeatletter
    \def\maxwidth{\ifdim\Gin@nat@width>\linewidth\linewidth
    \else\Gin@nat@width\fi}
    \makeatother
    \let\Oldincludegraphics\includegraphics
    % Set max figure width to be 80% of text width, for now hardcoded.
    \renewcommand{\includegraphics}[1]{\Oldincludegraphics[width=.8\maxwidth]{#1}}
    % Ensure that by default, figures have no caption (until we provide a
    % proper Figure object with a Caption API and a way to capture that
    % in the conversion process - todo).
    \usepackage{caption}
    \DeclareCaptionLabelFormat{nolabel}{}
    \captionsetup{labelformat=nolabel}

    \usepackage{adjustbox} % Used to constrain images to a maximum size 
    \usepackage{xcolor} % Allow colors to be defined
    \usepackage{enumerate} % Needed for markdown enumerations to work
    \usepackage{geometry} % Used to adjust the document margins
    \usepackage{amsmath} % Equations
    \usepackage{amssymb} % Equations
    \usepackage{textcomp} % defines textquotesingle
    % Hack from http://tex.stackexchange.com/a/47451/13684:
    \AtBeginDocument{%
        \def\PYZsq{\textquotesingle}% Upright quotes in Pygmentized code
    }
    \usepackage{upquote} % Upright quotes for verbatim code
    \usepackage{eurosym} % defines \euro
    \usepackage[mathletters]{ucs} % Extended unicode (utf-8) support
    \usepackage[utf8x]{inputenc} % Allow utf-8 characters in the tex document
    \usepackage{fancyvrb} % verbatim replacement that allows latex
    \usepackage{grffile} % extends the file name processing of package graphics 
                         % to support a larger range 
    % The hyperref package gives us a pdf with properly built
    % internal navigation ('pdf bookmarks' for the table of contents,
    % internal cross-reference links, web links for URLs, etc.)
    \usepackage{hyperref}
    \usepackage{longtable} % longtable support required by pandoc >1.10
    \usepackage{booktabs}  % table support for pandoc > 1.12.2
    \usepackage[inline]{enumitem} % IRkernel/repr support (it uses the enumerate* environment)
    \usepackage[normalem]{ulem} % ulem is needed to support strikethroughs (\sout)
                                % normalem makes italics be italics, not underlines
    

    
    
    % Colors for the hyperref package
    \definecolor{urlcolor}{rgb}{0,.145,.698}
    \definecolor{linkcolor}{rgb}{.71,0.21,0.01}
    \definecolor{citecolor}{rgb}{.12,.54,.11}

    % ANSI colors
    \definecolor{ansi-black}{HTML}{3E424D}
    \definecolor{ansi-black-intense}{HTML}{282C36}
    \definecolor{ansi-red}{HTML}{E75C58}
    \definecolor{ansi-red-intense}{HTML}{B22B31}
    \definecolor{ansi-green}{HTML}{00A250}
    \definecolor{ansi-green-intense}{HTML}{007427}
    \definecolor{ansi-yellow}{HTML}{DDB62B}
    \definecolor{ansi-yellow-intense}{HTML}{B27D12}
    \definecolor{ansi-blue}{HTML}{208FFB}
    \definecolor{ansi-blue-intense}{HTML}{0065CA}
    \definecolor{ansi-magenta}{HTML}{D160C4}
    \definecolor{ansi-magenta-intense}{HTML}{A03196}
    \definecolor{ansi-cyan}{HTML}{60C6C8}
    \definecolor{ansi-cyan-intense}{HTML}{258F8F}
    \definecolor{ansi-white}{HTML}{C5C1B4}
    \definecolor{ansi-white-intense}{HTML}{A1A6B2}

    % commands and environments needed by pandoc snippets
    % extracted from the output of `pandoc -s`
    \providecommand{\tightlist}{%
      \setlength{\itemsep}{0pt}\setlength{\parskip}{0pt}}
    \DefineVerbatimEnvironment{Highlighting}{Verbatim}{commandchars=\\\{\}}
    % Add ',fontsize=\small' for more characters per line
    \newenvironment{Shaded}{}{}
    \newcommand{\KeywordTok}[1]{\textcolor[rgb]{0.00,0.44,0.13}{\textbf{{#1}}}}
    \newcommand{\DataTypeTok}[1]{\textcolor[rgb]{0.56,0.13,0.00}{{#1}}}
    \newcommand{\DecValTok}[1]{\textcolor[rgb]{0.25,0.63,0.44}{{#1}}}
    \newcommand{\BaseNTok}[1]{\textcolor[rgb]{0.25,0.63,0.44}{{#1}}}
    \newcommand{\FloatTok}[1]{\textcolor[rgb]{0.25,0.63,0.44}{{#1}}}
    \newcommand{\CharTok}[1]{\textcolor[rgb]{0.25,0.44,0.63}{{#1}}}
    \newcommand{\StringTok}[1]{\textcolor[rgb]{0.25,0.44,0.63}{{#1}}}
    \newcommand{\CommentTok}[1]{\textcolor[rgb]{0.38,0.63,0.69}{\textit{{#1}}}}
    \newcommand{\OtherTok}[1]{\textcolor[rgb]{0.00,0.44,0.13}{{#1}}}
    \newcommand{\AlertTok}[1]{\textcolor[rgb]{1.00,0.00,0.00}{\textbf{{#1}}}}
    \newcommand{\FunctionTok}[1]{\textcolor[rgb]{0.02,0.16,0.49}{{#1}}}
    \newcommand{\RegionMarkerTok}[1]{{#1}}
    \newcommand{\ErrorTok}[1]{\textcolor[rgb]{1.00,0.00,0.00}{\textbf{{#1}}}}
    \newcommand{\NormalTok}[1]{{#1}}
    
    % Additional commands for more recent versions of Pandoc
    \newcommand{\ConstantTok}[1]{\textcolor[rgb]{0.53,0.00,0.00}{{#1}}}
    \newcommand{\SpecialCharTok}[1]{\textcolor[rgb]{0.25,0.44,0.63}{{#1}}}
    \newcommand{\VerbatimStringTok}[1]{\textcolor[rgb]{0.25,0.44,0.63}{{#1}}}
    \newcommand{\SpecialStringTok}[1]{\textcolor[rgb]{0.73,0.40,0.53}{{#1}}}
    \newcommand{\ImportTok}[1]{{#1}}
    \newcommand{\DocumentationTok}[1]{\textcolor[rgb]{0.73,0.13,0.13}{\textit{{#1}}}}
    \newcommand{\AnnotationTok}[1]{\textcolor[rgb]{0.38,0.63,0.69}{\textbf{\textit{{#1}}}}}
    \newcommand{\CommentVarTok}[1]{\textcolor[rgb]{0.38,0.63,0.69}{\textbf{\textit{{#1}}}}}
    \newcommand{\VariableTok}[1]{\textcolor[rgb]{0.10,0.09,0.49}{{#1}}}
    \newcommand{\ControlFlowTok}[1]{\textcolor[rgb]{0.00,0.44,0.13}{\textbf{{#1}}}}
    \newcommand{\OperatorTok}[1]{\textcolor[rgb]{0.40,0.40,0.40}{{#1}}}
    \newcommand{\BuiltInTok}[1]{{#1}}
    \newcommand{\ExtensionTok}[1]{{#1}}
    \newcommand{\PreprocessorTok}[1]{\textcolor[rgb]{0.74,0.48,0.00}{{#1}}}
    \newcommand{\AttributeTok}[1]{\textcolor[rgb]{0.49,0.56,0.16}{{#1}}}
    \newcommand{\InformationTok}[1]{\textcolor[rgb]{0.38,0.63,0.69}{\textbf{\textit{{#1}}}}}
    \newcommand{\WarningTok}[1]{\textcolor[rgb]{0.38,0.63,0.69}{\textbf{\textit{{#1}}}}}
    
    
    % Define a nice break command that doesn't care if a line doesn't already
    % exist.
    \def\br{\hspace*{\fill} \\* }
    % Math Jax compatability definitions
    \def\gt{>}
    \def\lt{<}
    % Document parameters
    \title{P1}
    
    
    

    % Pygments definitions
    
\makeatletter
\def\PY@reset{\let\PY@it=\relax \let\PY@bf=\relax%
    \let\PY@ul=\relax \let\PY@tc=\relax%
    \let\PY@bc=\relax \let\PY@ff=\relax}
\def\PY@tok#1{\csname PY@tok@#1\endcsname}
\def\PY@toks#1+{\ifx\relax#1\empty\else%
    \PY@tok{#1}\expandafter\PY@toks\fi}
\def\PY@do#1{\PY@bc{\PY@tc{\PY@ul{%
    \PY@it{\PY@bf{\PY@ff{#1}}}}}}}
\def\PY#1#2{\PY@reset\PY@toks#1+\relax+\PY@do{#2}}

\expandafter\def\csname PY@tok@cs\endcsname{\let\PY@it=\textit\def\PY@tc##1{\textcolor[rgb]{0.25,0.50,0.50}{##1}}}
\expandafter\def\csname PY@tok@nt\endcsname{\let\PY@bf=\textbf\def\PY@tc##1{\textcolor[rgb]{0.00,0.50,0.00}{##1}}}
\expandafter\def\csname PY@tok@vi\endcsname{\def\PY@tc##1{\textcolor[rgb]{0.10,0.09,0.49}{##1}}}
\expandafter\def\csname PY@tok@vg\endcsname{\def\PY@tc##1{\textcolor[rgb]{0.10,0.09,0.49}{##1}}}
\expandafter\def\csname PY@tok@c1\endcsname{\let\PY@it=\textit\def\PY@tc##1{\textcolor[rgb]{0.25,0.50,0.50}{##1}}}
\expandafter\def\csname PY@tok@err\endcsname{\def\PY@bc##1{\setlength{\fboxsep}{0pt}\fcolorbox[rgb]{1.00,0.00,0.00}{1,1,1}{\strut ##1}}}
\expandafter\def\csname PY@tok@nl\endcsname{\def\PY@tc##1{\textcolor[rgb]{0.63,0.63,0.00}{##1}}}
\expandafter\def\csname PY@tok@go\endcsname{\def\PY@tc##1{\textcolor[rgb]{0.53,0.53,0.53}{##1}}}
\expandafter\def\csname PY@tok@vc\endcsname{\def\PY@tc##1{\textcolor[rgb]{0.10,0.09,0.49}{##1}}}
\expandafter\def\csname PY@tok@s\endcsname{\def\PY@tc##1{\textcolor[rgb]{0.73,0.13,0.13}{##1}}}
\expandafter\def\csname PY@tok@se\endcsname{\let\PY@bf=\textbf\def\PY@tc##1{\textcolor[rgb]{0.73,0.40,0.13}{##1}}}
\expandafter\def\csname PY@tok@cpf\endcsname{\let\PY@it=\textit\def\PY@tc##1{\textcolor[rgb]{0.25,0.50,0.50}{##1}}}
\expandafter\def\csname PY@tok@sd\endcsname{\let\PY@it=\textit\def\PY@tc##1{\textcolor[rgb]{0.73,0.13,0.13}{##1}}}
\expandafter\def\csname PY@tok@ne\endcsname{\let\PY@bf=\textbf\def\PY@tc##1{\textcolor[rgb]{0.82,0.25,0.23}{##1}}}
\expandafter\def\csname PY@tok@na\endcsname{\def\PY@tc##1{\textcolor[rgb]{0.49,0.56,0.16}{##1}}}
\expandafter\def\csname PY@tok@gs\endcsname{\let\PY@bf=\textbf}
\expandafter\def\csname PY@tok@mh\endcsname{\def\PY@tc##1{\textcolor[rgb]{0.40,0.40,0.40}{##1}}}
\expandafter\def\csname PY@tok@bp\endcsname{\def\PY@tc##1{\textcolor[rgb]{0.00,0.50,0.00}{##1}}}
\expandafter\def\csname PY@tok@sx\endcsname{\def\PY@tc##1{\textcolor[rgb]{0.00,0.50,0.00}{##1}}}
\expandafter\def\csname PY@tok@gh\endcsname{\let\PY@bf=\textbf\def\PY@tc##1{\textcolor[rgb]{0.00,0.00,0.50}{##1}}}
\expandafter\def\csname PY@tok@sb\endcsname{\def\PY@tc##1{\textcolor[rgb]{0.73,0.13,0.13}{##1}}}
\expandafter\def\csname PY@tok@gr\endcsname{\def\PY@tc##1{\textcolor[rgb]{1.00,0.00,0.00}{##1}}}
\expandafter\def\csname PY@tok@kt\endcsname{\def\PY@tc##1{\textcolor[rgb]{0.69,0.00,0.25}{##1}}}
\expandafter\def\csname PY@tok@il\endcsname{\def\PY@tc##1{\textcolor[rgb]{0.40,0.40,0.40}{##1}}}
\expandafter\def\csname PY@tok@nn\endcsname{\let\PY@bf=\textbf\def\PY@tc##1{\textcolor[rgb]{0.00,0.00,1.00}{##1}}}
\expandafter\def\csname PY@tok@w\endcsname{\def\PY@tc##1{\textcolor[rgb]{0.73,0.73,0.73}{##1}}}
\expandafter\def\csname PY@tok@mb\endcsname{\def\PY@tc##1{\textcolor[rgb]{0.40,0.40,0.40}{##1}}}
\expandafter\def\csname PY@tok@sr\endcsname{\def\PY@tc##1{\textcolor[rgb]{0.73,0.40,0.53}{##1}}}
\expandafter\def\csname PY@tok@fm\endcsname{\def\PY@tc##1{\textcolor[rgb]{0.00,0.00,1.00}{##1}}}
\expandafter\def\csname PY@tok@vm\endcsname{\def\PY@tc##1{\textcolor[rgb]{0.10,0.09,0.49}{##1}}}
\expandafter\def\csname PY@tok@kc\endcsname{\let\PY@bf=\textbf\def\PY@tc##1{\textcolor[rgb]{0.00,0.50,0.00}{##1}}}
\expandafter\def\csname PY@tok@mo\endcsname{\def\PY@tc##1{\textcolor[rgb]{0.40,0.40,0.40}{##1}}}
\expandafter\def\csname PY@tok@cp\endcsname{\def\PY@tc##1{\textcolor[rgb]{0.74,0.48,0.00}{##1}}}
\expandafter\def\csname PY@tok@ni\endcsname{\let\PY@bf=\textbf\def\PY@tc##1{\textcolor[rgb]{0.60,0.60,0.60}{##1}}}
\expandafter\def\csname PY@tok@gi\endcsname{\def\PY@tc##1{\textcolor[rgb]{0.00,0.63,0.00}{##1}}}
\expandafter\def\csname PY@tok@nd\endcsname{\def\PY@tc##1{\textcolor[rgb]{0.67,0.13,1.00}{##1}}}
\expandafter\def\csname PY@tok@mi\endcsname{\def\PY@tc##1{\textcolor[rgb]{0.40,0.40,0.40}{##1}}}
\expandafter\def\csname PY@tok@o\endcsname{\def\PY@tc##1{\textcolor[rgb]{0.40,0.40,0.40}{##1}}}
\expandafter\def\csname PY@tok@kr\endcsname{\let\PY@bf=\textbf\def\PY@tc##1{\textcolor[rgb]{0.00,0.50,0.00}{##1}}}
\expandafter\def\csname PY@tok@c\endcsname{\let\PY@it=\textit\def\PY@tc##1{\textcolor[rgb]{0.25,0.50,0.50}{##1}}}
\expandafter\def\csname PY@tok@mf\endcsname{\def\PY@tc##1{\textcolor[rgb]{0.40,0.40,0.40}{##1}}}
\expandafter\def\csname PY@tok@cm\endcsname{\let\PY@it=\textit\def\PY@tc##1{\textcolor[rgb]{0.25,0.50,0.50}{##1}}}
\expandafter\def\csname PY@tok@sc\endcsname{\def\PY@tc##1{\textcolor[rgb]{0.73,0.13,0.13}{##1}}}
\expandafter\def\csname PY@tok@si\endcsname{\let\PY@bf=\textbf\def\PY@tc##1{\textcolor[rgb]{0.73,0.40,0.53}{##1}}}
\expandafter\def\csname PY@tok@nc\endcsname{\let\PY@bf=\textbf\def\PY@tc##1{\textcolor[rgb]{0.00,0.00,1.00}{##1}}}
\expandafter\def\csname PY@tok@nv\endcsname{\def\PY@tc##1{\textcolor[rgb]{0.10,0.09,0.49}{##1}}}
\expandafter\def\csname PY@tok@kd\endcsname{\let\PY@bf=\textbf\def\PY@tc##1{\textcolor[rgb]{0.00,0.50,0.00}{##1}}}
\expandafter\def\csname PY@tok@sa\endcsname{\def\PY@tc##1{\textcolor[rgb]{0.73,0.13,0.13}{##1}}}
\expandafter\def\csname PY@tok@gd\endcsname{\def\PY@tc##1{\textcolor[rgb]{0.63,0.00,0.00}{##1}}}
\expandafter\def\csname PY@tok@k\endcsname{\let\PY@bf=\textbf\def\PY@tc##1{\textcolor[rgb]{0.00,0.50,0.00}{##1}}}
\expandafter\def\csname PY@tok@dl\endcsname{\def\PY@tc##1{\textcolor[rgb]{0.73,0.13,0.13}{##1}}}
\expandafter\def\csname PY@tok@kp\endcsname{\def\PY@tc##1{\textcolor[rgb]{0.00,0.50,0.00}{##1}}}
\expandafter\def\csname PY@tok@nf\endcsname{\def\PY@tc##1{\textcolor[rgb]{0.00,0.00,1.00}{##1}}}
\expandafter\def\csname PY@tok@gu\endcsname{\let\PY@bf=\textbf\def\PY@tc##1{\textcolor[rgb]{0.50,0.00,0.50}{##1}}}
\expandafter\def\csname PY@tok@s1\endcsname{\def\PY@tc##1{\textcolor[rgb]{0.73,0.13,0.13}{##1}}}
\expandafter\def\csname PY@tok@s2\endcsname{\def\PY@tc##1{\textcolor[rgb]{0.73,0.13,0.13}{##1}}}
\expandafter\def\csname PY@tok@ge\endcsname{\let\PY@it=\textit}
\expandafter\def\csname PY@tok@nb\endcsname{\def\PY@tc##1{\textcolor[rgb]{0.00,0.50,0.00}{##1}}}
\expandafter\def\csname PY@tok@ch\endcsname{\let\PY@it=\textit\def\PY@tc##1{\textcolor[rgb]{0.25,0.50,0.50}{##1}}}
\expandafter\def\csname PY@tok@ss\endcsname{\def\PY@tc##1{\textcolor[rgb]{0.10,0.09,0.49}{##1}}}
\expandafter\def\csname PY@tok@kn\endcsname{\let\PY@bf=\textbf\def\PY@tc##1{\textcolor[rgb]{0.00,0.50,0.00}{##1}}}
\expandafter\def\csname PY@tok@no\endcsname{\def\PY@tc##1{\textcolor[rgb]{0.53,0.00,0.00}{##1}}}
\expandafter\def\csname PY@tok@gp\endcsname{\let\PY@bf=\textbf\def\PY@tc##1{\textcolor[rgb]{0.00,0.00,0.50}{##1}}}
\expandafter\def\csname PY@tok@m\endcsname{\def\PY@tc##1{\textcolor[rgb]{0.40,0.40,0.40}{##1}}}
\expandafter\def\csname PY@tok@ow\endcsname{\let\PY@bf=\textbf\def\PY@tc##1{\textcolor[rgb]{0.67,0.13,1.00}{##1}}}
\expandafter\def\csname PY@tok@sh\endcsname{\def\PY@tc##1{\textcolor[rgb]{0.73,0.13,0.13}{##1}}}
\expandafter\def\csname PY@tok@gt\endcsname{\def\PY@tc##1{\textcolor[rgb]{0.00,0.27,0.87}{##1}}}

\def\PYZbs{\char`\\}
\def\PYZus{\char`\_}
\def\PYZob{\char`\{}
\def\PYZcb{\char`\}}
\def\PYZca{\char`\^}
\def\PYZam{\char`\&}
\def\PYZlt{\char`\<}
\def\PYZgt{\char`\>}
\def\PYZsh{\char`\#}
\def\PYZpc{\char`\%}
\def\PYZdl{\char`\$}
\def\PYZhy{\char`\-}
\def\PYZsq{\char`\'}
\def\PYZdq{\char`\"}
\def\PYZti{\char`\~}
% for compatibility with earlier versions
\def\PYZat{@}
\def\PYZlb{[}
\def\PYZrb{]}
\makeatother


    % Exact colors from NB
    \definecolor{incolor}{rgb}{0.0, 0.0, 0.5}
    \definecolor{outcolor}{rgb}{0.545, 0.0, 0.0}



    
    % Prevent overflowing lines due to hard-to-break entities
    \sloppy 
    % Setup hyperref package
    \hypersetup{
      breaklinks=true,  % so long urls are correctly broken across lines
      colorlinks=true,
      urlcolor=urlcolor,
      linkcolor=linkcolor,
      citecolor=citecolor,
      }
    % Slightly bigger margins than the latex defaults
    
    \geometry{verbose,tmargin=1in,bmargin=1in,lmargin=1in,rmargin=1in}
    
    

    \begin{document}
    
    
    \maketitle
    
    

    
    \hypertarget{self-driving-car-engineer-nanodegree}{%
\section{Self-Driving Car Engineer
Nanodegree}\label{self-driving-car-engineer-nanodegree}}

\hypertarget{project-finding-lane-lines-on-the-road}{%
\subsection{\texorpdfstring{Project: \textbf{Finding Lane Lines on the
Road}}{Project: Finding Lane Lines on the Road}}\label{project-finding-lane-lines-on-the-road}}

\begin{center}\rule{0.5\linewidth}{\linethickness}\end{center}

In this project, you will use the tools you learned about in the lesson
to identify lane lines on the road. You can develop your pipeline on a
series of individual images, and later apply the result to a video
stream (really just a series of images). Check out the video clip
``raw-lines-example.mp4'' (also contained in this repository) to see
what the output should look like after using the helper functions below.

Once you have a result that looks roughly like
``raw-lines-example.mp4'', you'll need to get creative and try to
average and/or extrapolate the line segments you've detected to map out
the full extent of the lane lines. You can see an example of the result
you're going for in the video ``P1\_example.mp4''. Ultimately, you would
like to draw just one line for the left side of the lane, and one for
the right.

In addition to implementing code, there is a brief writeup to complete.
The writeup should be completed in a separate file, which can be either
a markdown file or a pdf document. There is a
\href{https://github.com/udacity/CarND-LaneLines-P1/blob/master/writeup_template.md}{write
up template} that can be used to guide the writing process. Completing
both the code in the Ipython notebook and the writeup template will
cover all of the
\href{https://review.udacity.com/\#!/rubrics/322/view}{rubric points}
for this project.

    \textbf{The tools you have are color selection, region of interest
selection, grayscaling, Gaussian smoothing, Canny Edge Detection and
Hough Tranform line detection. You are also free to explore and try
other techniques that were not presented in the lesson. Your goal is
piece together a pipeline to detect the line segments in the image, then
average/extrapolate them and draw them onto the image for display (as
below). Once you have a working pipeline, try it out on the video stream
below.}

\begin{center}\rule{0.5\linewidth}{\linethickness}\end{center}

Your output should look something like this (above) after detecting line
segments using the helper functions below

Your goal is to connect/average/extrapolate line segments to get output
like this

    \textbf{Run the cell below to import some packages. If you get an
\texttt{import\ error} for a package you've already installed, try
changing your kernel (select the Kernel menu above --\textgreater{}
Change Kernel). Still have problems? Try relaunching Jupyter Notebook
from the terminal prompt. Also, consult the forums for more
troubleshooting tips.}

    \hypertarget{import-packages}{%
\subsection{Import Packages}\label{import-packages}}

    \begin{Verbatim}[commandchars=\\\{\}]
{\color{incolor}In [{\color{incolor}1}]:} \PY{c+c1}{\PYZsh{}importing some useful packages}
        \PY{k+kn}{import} \PY{n+nn}{matplotlib}\PY{n+nn}{.}\PY{n+nn}{pyplot} \PY{k}{as} \PY{n+nn}{plt}
        \PY{k+kn}{import} \PY{n+nn}{matplotlib}\PY{n+nn}{.}\PY{n+nn}{image} \PY{k}{as} \PY{n+nn}{mpimg}
        \PY{k+kn}{import} \PY{n+nn}{numpy} \PY{k}{as} \PY{n+nn}{np}
        \PY{k+kn}{import} \PY{n+nn}{cv2}
        \PY{o}{\PYZpc{}}\PY{k}{matplotlib} inline
\end{Verbatim}


    \hypertarget{read-in-an-image}{%
\subsection{Read in an Image}\label{read-in-an-image}}

    \begin{Verbatim}[commandchars=\\\{\}]
{\color{incolor}In [{\color{incolor}2}]:} \PY{c+c1}{\PYZsh{}reading in an image}
        \PY{n}{image} \PY{o}{=} \PY{n}{mpimg}\PY{o}{.}\PY{n}{imread}\PY{p}{(}\PY{l+s+s1}{\PYZsq{}}\PY{l+s+s1}{test\PYZus{}images/solidWhiteRight.jpg}\PY{l+s+s1}{\PYZsq{}}\PY{p}{)}
        
        \PY{c+c1}{\PYZsh{}printing out some stats and plotting}
        \PY{n+nb}{print}\PY{p}{(}\PY{l+s+s1}{\PYZsq{}}\PY{l+s+s1}{This image is:}\PY{l+s+s1}{\PYZsq{}}\PY{p}{,} \PY{n+nb}{type}\PY{p}{(}\PY{n}{image}\PY{p}{)}\PY{p}{,} \PY{l+s+s1}{\PYZsq{}}\PY{l+s+s1}{with dimensions:}\PY{l+s+s1}{\PYZsq{}}\PY{p}{,} \PY{n}{image}\PY{o}{.}\PY{n}{shape}\PY{p}{)}
        \PY{n}{plt}\PY{o}{.}\PY{n}{imshow}\PY{p}{(}\PY{n}{image}\PY{p}{)}  \PY{c+c1}{\PYZsh{} if you wanted to show a single color channel image called \PYZsq{}gray\PYZsq{}, for example, call as plt.imshow(gray, cmap=\PYZsq{}gray\PYZsq{})}
\end{Verbatim}


    \begin{Verbatim}[commandchars=\\\{\}]
This image is: <class 'numpy.ndarray'> with dimensions: (540, 960, 3)

    \end{Verbatim}

\begin{Verbatim}[commandchars=\\\{\}]
{\color{outcolor}Out[{\color{outcolor}2}]:} <matplotlib.image.AxesImage at 0x121c45e10>
\end{Verbatim}
            
    \begin{center}
    \adjustimage{max size={0.9\linewidth}{0.9\paperheight}}{output_6_2.png}
    \end{center}
    { \hspace*{\fill} \\}
    
    \hypertarget{ideas-for-lane-detection-pipeline}{%
\subsection{Ideas for Lane Detection
Pipeline}\label{ideas-for-lane-detection-pipeline}}

    \textbf{Some OpenCV functions (beyond those introduced in the lesson)
that might be useful for this project are:}

\texttt{cv2.inRange()} for color selection\\
\texttt{cv2.fillPoly()} for regions selection\\
\texttt{cv2.line()} to draw lines on an image given endpoints\\
\texttt{cv2.addWeighted()} to coadd / overlay two images
\texttt{cv2.cvtColor()} to grayscale or change color
\texttt{cv2.imwrite()} to output images to file\\
\texttt{cv2.bitwise\_and()} to apply a mask to an image

\textbf{Check out the OpenCV documentation to learn about these and
discover even more awesome functionality!}

    \hypertarget{helper-functions}{%
\subsection{Helper Functions}\label{helper-functions}}

    Below are some helper functions to help get you started. They should
look familiar from the lesson!

    \begin{Verbatim}[commandchars=\\\{\}]
{\color{incolor}In [{\color{incolor}3}]:} \PY{k}{def} \PY{n+nf}{slope\PYZus{}intercept}\PY{p}{(}\PY{n}{x1}\PY{p}{,} \PY{n}{y1}\PY{p}{,} \PY{n}{x2}\PY{p}{,} \PY{n}{y2}\PY{p}{)}\PY{p}{:}
            \PY{l+s+sd}{\PYZdq{}\PYZdq{}\PYZdq{}}
        \PY{l+s+sd}{    This function returns slope and intercpt given two points}
        \PY{l+s+sd}{    \PYZdq{}\PYZdq{}\PYZdq{}}
            \PY{n}{m} \PY{o}{=} \PY{p}{(}\PY{n}{y2} \PY{o}{\PYZhy{}} \PY{n}{y1}\PY{p}{)}\PY{o}{/}\PY{p}{(}\PY{n}{x2} \PY{o}{\PYZhy{}} \PY{n}{x1}\PY{p}{)}
            \PY{n}{c} \PY{o}{=} \PY{n}{y1} \PY{o}{\PYZhy{}} \PY{p}{(}\PY{n}{m}\PY{o}{*}\PY{n}{x1}\PY{p}{)}
            \PY{k}{return} \PY{n}{m}\PY{p}{,} \PY{n}{c}
\end{Verbatim}


    \begin{Verbatim}[commandchars=\\\{\}]
{\color{incolor}In [{\color{incolor}4}]:} \PY{k}{def} \PY{n+nf}{average\PYZus{}lines}\PY{p}{(}\PY{n}{lines}\PY{p}{)}\PY{p}{:}
            \PY{l+s+sd}{\PYZdq{}\PYZdq{}\PYZdq{}}
        \PY{l+s+sd}{    This function receives a set of lines(each line consist of two set of points) and}
        \PY{l+s+sd}{    returns two averaged out line based on the slope and intercept.}
        \PY{l+s+sd}{    }
        \PY{l+s+sd}{    Average out slope and intercept together when slope is less than zero and when }
        \PY{l+s+sd}{    slope is greater than zero.}
        \PY{l+s+sd}{    Ignores the line when slope is infinity.}
        \PY{l+s+sd}{    \PYZdq{}\PYZdq{}\PYZdq{}}
            
            \PY{c+c1}{\PYZsh{} Dictionary for positive and negative slopes and intercepSts}
            \PY{n}{positives} \PY{o}{=} \PY{p}{\PYZob{}}\PY{l+s+s2}{\PYZdq{}}\PY{l+s+s2}{slope}\PY{l+s+s2}{\PYZdq{}}\PY{p}{:} \PY{p}{[}\PY{p}{]}\PY{p}{,} \PY{l+s+s2}{\PYZdq{}}\PY{l+s+s2}{intercept}\PY{l+s+s2}{\PYZdq{}}\PY{p}{:} \PY{p}{[}\PY{p}{]}\PY{p}{\PYZcb{}}
            \PY{n}{negatives} \PY{o}{=} \PY{p}{\PYZob{}}\PY{l+s+s2}{\PYZdq{}}\PY{l+s+s2}{slope}\PY{l+s+s2}{\PYZdq{}}\PY{p}{:} \PY{p}{[}\PY{p}{]}\PY{p}{,} \PY{l+s+s2}{\PYZdq{}}\PY{l+s+s2}{intercept}\PY{l+s+s2}{\PYZdq{}}\PY{p}{:} \PY{p}{[}\PY{p}{]}\PY{p}{\PYZcb{}}
            \PY{k}{for} \PY{n}{line} \PY{o+ow}{in} \PY{n}{lines}\PY{p}{:}
                \PY{k}{for} \PY{n}{x1}\PY{p}{,} \PY{n}{y1}\PY{p}{,} \PY{n}{x2}\PY{p}{,} \PY{n}{y2} \PY{o+ow}{in} \PY{n}{line}\PY{p}{:}
                    \PY{n}{m}\PY{p}{,} \PY{n}{c} \PY{o}{=} \PY{n}{slope\PYZus{}intercept}\PY{p}{(}\PY{n}{x1}\PY{p}{,} \PY{n}{y1}\PY{p}{,} \PY{n}{x2}\PY{p}{,} \PY{n}{y2}\PY{p}{)}
                    
                    \PY{c+c1}{\PYZsh{} Skip when slope is infinity}
                    \PY{k}{if} \PY{n}{np}\PY{o}{.}\PY{n}{isnan}\PY{p}{(}\PY{n}{m}\PY{p}{)}\PY{p}{:}
                        \PY{k}{continue}
                        
                    \PY{k}{if} \PY{n}{m}\PY{o}{\PYZlt{}}\PY{l+m+mi}{0}\PY{p}{:}
                        \PY{n}{negatives}\PY{p}{[}\PY{l+s+s1}{\PYZsq{}}\PY{l+s+s1}{slope}\PY{l+s+s1}{\PYZsq{}}\PY{p}{]}\PY{o}{.}\PY{n}{append}\PY{p}{(}\PY{n}{m}\PY{p}{)}
                        \PY{n}{negatives}\PY{p}{[}\PY{l+s+s1}{\PYZsq{}}\PY{l+s+s1}{intercept}\PY{l+s+s1}{\PYZsq{}}\PY{p}{]}\PY{o}{.}\PY{n}{append}\PY{p}{(}\PY{n}{c}\PY{p}{)}
                    \PY{k}{else}\PY{p}{:}
                        \PY{n}{positives}\PY{p}{[}\PY{l+s+s1}{\PYZsq{}}\PY{l+s+s1}{slope}\PY{l+s+s1}{\PYZsq{}}\PY{p}{]}\PY{o}{.}\PY{n}{append}\PY{p}{(}\PY{n}{m}\PY{p}{)}
                        \PY{n}{positives}\PY{p}{[}\PY{l+s+s1}{\PYZsq{}}\PY{l+s+s1}{intercept}\PY{l+s+s1}{\PYZsq{}}\PY{p}{]}\PY{o}{.}\PY{n}{append}\PY{p}{(}\PY{n}{c}\PY{p}{)}
            
            \PY{n}{new\PYZus{}lines} \PY{o}{=} \PY{p}{[}\PY{p}{]}
            
            \PY{c+c1}{\PYZsh{} Checks if these is at least one slope(otherwise slope would be infinity)     }
            \PY{k}{if} \PY{n+nb}{len}\PY{p}{(}\PY{n}{positives}\PY{p}{[}\PY{l+s+s1}{\PYZsq{}}\PY{l+s+s1}{slope}\PY{l+s+s1}{\PYZsq{}}\PY{p}{]}\PY{p}{)} \PY{o}{\PYZgt{}} \PY{l+m+mi}{0}\PY{p}{:}
                \PY{n}{m1} \PY{o}{=} \PY{n}{np}\PY{o}{.}\PY{n}{mean}\PY{p}{(}\PY{n}{positives}\PY{p}{[}\PY{l+s+s1}{\PYZsq{}}\PY{l+s+s1}{slope}\PY{l+s+s1}{\PYZsq{}}\PY{p}{]}\PY{p}{)}
                \PY{n}{c1} \PY{o}{=} \PY{n}{np}\PY{o}{.}\PY{n}{mean}\PY{p}{(}\PY{n}{positives}\PY{p}{[}\PY{l+s+s1}{\PYZsq{}}\PY{l+s+s1}{intercept}\PY{l+s+s1}{\PYZsq{}}\PY{p}{]}\PY{p}{)}
                
                \PY{n}{xs} \PY{o}{=} \PY{n}{np}\PY{o}{.}\PY{n}{arange}\PY{p}{(}\PY{l+m+mi}{500}\PY{p}{,} \PY{l+m+mi}{960}\PY{p}{,} \PY{l+m+mi}{1}\PY{p}{)}
                \PY{k}{for} \PY{n}{i} \PY{o+ow}{in} \PY{n+nb}{range}\PY{p}{(}\PY{n+nb}{len}\PY{p}{(}\PY{n}{xs}\PY{p}{)}\PY{o}{\PYZhy{}}\PY{l+m+mi}{1}\PY{p}{)}\PY{p}{:}
                    \PY{n}{x1} \PY{o}{=} \PY{n}{xs}\PY{p}{[}\PY{n}{i}\PY{p}{]}
                    \PY{n}{y1} \PY{o}{=} \PY{n}{x1} \PY{o}{*} \PY{n}{m1} \PY{o}{+} \PY{n}{c1}
                    \PY{n}{x2} \PY{o}{=} \PY{n}{xs}\PY{p}{[}\PY{n}{i} \PY{o}{+} \PY{l+m+mi}{1}\PY{p}{]}
                    \PY{n}{y2} \PY{o}{=} \PY{n}{x2} \PY{o}{*} \PY{n}{m1} \PY{o}{+} \PY{n}{c1}
                    \PY{n}{new\PYZus{}lines}\PY{o}{.}\PY{n}{append}\PY{p}{(}\PY{p}{[}\PY{p}{[}\PY{n}{x1}\PY{p}{,} \PY{n}{y1}\PY{p}{,} \PY{n}{x2}\PY{p}{,} \PY{n}{y2}\PY{p}{]}\PY{p}{]}\PY{p}{)}
                
                
            \PY{k}{if} \PY{n+nb}{len}\PY{p}{(}\PY{n}{negatives}\PY{p}{[}\PY{l+s+s1}{\PYZsq{}}\PY{l+s+s1}{slope}\PY{l+s+s1}{\PYZsq{}}\PY{p}{]}\PY{p}{)} \PY{o}{\PYZgt{}} \PY{l+m+mi}{0}\PY{p}{:}
                \PY{n}{m2} \PY{o}{=} \PY{n}{np}\PY{o}{.}\PY{n}{mean}\PY{p}{(}\PY{n}{negatives}\PY{p}{[}\PY{l+s+s1}{\PYZsq{}}\PY{l+s+s1}{slope}\PY{l+s+s1}{\PYZsq{}}\PY{p}{]}\PY{p}{)}
                \PY{n}{c2} \PY{o}{=} \PY{n}{np}\PY{o}{.}\PY{n}{mean}\PY{p}{(}\PY{n}{negatives}\PY{p}{[}\PY{l+s+s1}{\PYZsq{}}\PY{l+s+s1}{intercept}\PY{l+s+s1}{\PYZsq{}}\PY{p}{]}\PY{p}{)}
                
                \PY{n}{xs} \PY{o}{=} \PY{n}{np}\PY{o}{.}\PY{n}{arange}\PY{p}{(}\PY{l+m+mi}{0}\PY{p}{,} \PY{l+m+mi}{480}\PY{p}{,} \PY{l+m+mi}{1}\PY{p}{)}
                \PY{k}{for} \PY{n}{i} \PY{o+ow}{in} \PY{n+nb}{range}\PY{p}{(}\PY{n+nb}{len}\PY{p}{(}\PY{n}{xs}\PY{p}{)}\PY{o}{\PYZhy{}}\PY{l+m+mi}{1}\PY{p}{)}\PY{p}{:}
                    \PY{n}{x1} \PY{o}{=} \PY{n}{xs}\PY{p}{[}\PY{n}{i}\PY{p}{]}
                    \PY{n}{y1} \PY{o}{=} \PY{n}{x1} \PY{o}{*} \PY{n}{m2} \PY{o}{+} \PY{n}{c2}
                    \PY{n}{x2} \PY{o}{=} \PY{n}{xs}\PY{p}{[}\PY{n}{i}\PY{o}{+}\PY{l+m+mi}{1}\PY{p}{]}
                    \PY{n}{y2} \PY{o}{=} \PY{n}{x2} \PY{o}{*} \PY{n}{m2} \PY{o}{+} \PY{n}{c2}
                    \PY{n}{new\PYZus{}lines}\PY{o}{.}\PY{n}{append}\PY{p}{(}\PY{p}{[}\PY{p}{[}\PY{n}{x1}\PY{p}{,} \PY{n}{y1}\PY{p}{,} \PY{n}{x2}\PY{p}{,} \PY{n}{y2}\PY{p}{]}\PY{p}{]}\PY{p}{)}
                
            \PY{k}{return} \PY{n}{np}\PY{o}{.}\PY{n}{array}\PY{p}{(}\PY{n}{new\PYZus{}lines}\PY{p}{)}\PY{o}{.}\PY{n}{astype}\PY{p}{(}\PY{n+nb}{int}\PY{p}{)}
\end{Verbatim}


    \begin{Verbatim}[commandchars=\\\{\}]
{\color{incolor}In [{\color{incolor}5}]:} \PY{k}{def} \PY{n+nf}{extend\PYZus{}line}\PY{p}{(}\PY{n}{lines}\PY{p}{)}\PY{p}{:}
            \PY{l+s+sd}{\PYZdq{}\PYZdq{}\PYZdq{}}
        \PY{l+s+sd}{    Given a set of lines, this function extends the line based}
        \PY{l+s+sd}{    on it\PYZsq{}s slope and intercepts. The genrated lines are not }
        \PY{l+s+sd}{    smooth. So lines are averaged out and returned.ss}
        \PY{l+s+sd}{    \PYZdq{}\PYZdq{}\PYZdq{}}
            \PY{n}{new\PYZus{}lines} \PY{o}{=} \PY{p}{[}\PY{p}{]}
            \PY{k}{for} \PY{n}{line} \PY{o+ow}{in} \PY{n}{lines}\PY{p}{:}
                \PY{k}{for} \PY{n}{x1}\PY{p}{,} \PY{n}{y1}\PY{p}{,} \PY{n}{x2}\PY{p}{,} \PY{n}{y2} \PY{o+ow}{in} \PY{n}{line}\PY{p}{:}
                    \PY{n}{m}\PY{p}{,} \PY{n}{c} \PY{o}{=} \PY{n}{slope\PYZus{}intercept}\PY{p}{(}\PY{n}{x1}\PY{p}{,} \PY{n}{y1}\PY{p}{,} \PY{n}{x2}\PY{p}{,} \PY{n}{y2}\PY{p}{)}
        
                    \PY{k}{if} \PY{n}{m} \PY{o}{\PYZgt{}}\PY{o}{=} \PY{l+m+mi}{0}\PY{p}{:}
                        \PY{n}{xs} \PY{o}{=} \PY{n}{np}\PY{o}{.}\PY{n}{arange}\PY{p}{(}\PY{l+m+mi}{0}\PY{p}{,} \PY{l+m+mi}{450}\PY{p}{,} \PY{l+m+mi}{5}\PY{p}{)}
                        \PY{k}{for} \PY{n}{x} \PY{o+ow}{in} \PY{n}{xs}\PY{p}{:}
                            \PY{n}{x3} \PY{o}{=} \PY{n}{x1} \PY{o}{+} \PY{n}{x}
                            \PY{n}{y3} \PY{o}{=} \PY{n}{m}\PY{o}{*}\PY{n}{x3} \PY{o}{+} \PY{n}{c}
                            \PY{n}{new\PYZus{}lines}\PY{o}{.}\PY{n}{append}\PY{p}{(}\PY{p}{[}\PY{p}{[}\PY{n}{x3}\PY{p}{,} \PY{n}{y3}\PY{p}{,} \PY{n}{x1}\PY{p}{,} \PY{n}{y1}\PY{p}{]}\PY{p}{]}\PY{p}{)}
                    \PY{k}{if} \PY{n}{m} \PY{o}{\PYZlt{}} \PY{l+m+mi}{0}\PY{p}{:}
                        \PY{n}{xs} \PY{o}{=} \PY{n}{np}\PY{o}{.}\PY{n}{arange}\PY{p}{(}\PY{l+m+mi}{0}\PY{p}{,} \PY{l+m+mi}{450}\PY{p}{,} \PY{l+m+mi}{5}\PY{p}{)}
                        \PY{k}{for} \PY{n}{x} \PY{o+ow}{in} \PY{n}{xs}\PY{p}{:}
                            \PY{n}{x3} \PY{o}{=} \PY{n}{x1} \PY{o}{\PYZhy{}} \PY{n}{x}
                            \PY{n}{y3} \PY{o}{=} \PY{n}{m}\PY{o}{*}\PY{n}{x3} \PY{o}{+} \PY{n}{c}
                            \PY{n}{new\PYZus{}lines}\PY{o}{.}\PY{n}{append}\PY{p}{(}\PY{p}{[}\PY{p}{[}\PY{n}{x3}\PY{p}{,} \PY{n}{y3}\PY{p}{,} \PY{n}{x2}\PY{p}{,} \PY{n}{y2}\PY{p}{]}\PY{p}{]}\PY{p}{)}
            \PY{n}{extended\PYZus{}lines} \PY{o}{=} \PY{n}{np}\PY{o}{.}\PY{n}{concatenate}\PY{p}{(}\PY{p}{(}\PY{n}{lines}\PY{p}{,} \PY{n}{np}\PY{o}{.}\PY{n}{array}\PY{p}{(}\PY{n}{new\PYZus{}lines}\PY{p}{)}\PY{o}{.}\PY{n}{astype}\PY{p}{(}\PY{n+nb}{int}\PY{p}{)}\PY{p}{)}\PY{p}{)}
            \PY{k}{return} \PY{n}{average\PYZus{}lines}\PY{p}{(}\PY{n}{extended\PYZus{}lines}\PY{p}{)}
\end{Verbatim}


    \begin{Verbatim}[commandchars=\\\{\}]
{\color{incolor}In [{\color{incolor}6}]:} \PY{k+kn}{import} \PY{n+nn}{math}
        
        \PY{k}{def} \PY{n+nf}{grayscale}\PY{p}{(}\PY{n}{img}\PY{p}{)}\PY{p}{:}
            \PY{l+s+sd}{\PYZdq{}\PYZdq{}\PYZdq{}Applies the Grayscale transform}
        \PY{l+s+sd}{    This will return an image with only one color channel}
        \PY{l+s+sd}{    but NOTE: to see the returned image as grayscale}
        \PY{l+s+sd}{    (assuming your grayscaled image is called \PYZsq{}gray\PYZsq{})}
        \PY{l+s+sd}{    you should call plt.imshow(gray, cmap=\PYZsq{}gray\PYZsq{})\PYZdq{}\PYZdq{}\PYZdq{}}
            \PY{k}{return} \PY{n}{cv2}\PY{o}{.}\PY{n}{cvtColor}\PY{p}{(}\PY{n}{img}\PY{p}{,} \PY{n}{cv2}\PY{o}{.}\PY{n}{COLOR\PYZus{}RGB2GRAY}\PY{p}{)}
            \PY{c+c1}{\PYZsh{} Or use BGR2GRAY if you read an image with cv2.imread()}
            \PY{c+c1}{\PYZsh{} return cv2.cvtColor(img, cv2.COLOR\PYZus{}BGR2GRAY)}
            
        \PY{k}{def} \PY{n+nf}{canny}\PY{p}{(}\PY{n}{img}\PY{p}{,} \PY{n}{low\PYZus{}threshold}\PY{p}{,} \PY{n}{high\PYZus{}threshold}\PY{p}{)}\PY{p}{:}
            \PY{l+s+sd}{\PYZdq{}\PYZdq{}\PYZdq{}Applies the Canny transform\PYZdq{}\PYZdq{}\PYZdq{}}
            \PY{k}{return} \PY{n}{cv2}\PY{o}{.}\PY{n}{Canny}\PY{p}{(}\PY{n}{img}\PY{p}{,} \PY{n}{low\PYZus{}threshold}\PY{p}{,} \PY{n}{high\PYZus{}threshold}\PY{p}{)}
        
        \PY{k}{def} \PY{n+nf}{gaussian\PYZus{}blur}\PY{p}{(}\PY{n}{img}\PY{p}{,} \PY{n}{kernel\PYZus{}size}\PY{p}{)}\PY{p}{:}
            \PY{l+s+sd}{\PYZdq{}\PYZdq{}\PYZdq{}Applies a Gaussian Noise kernel\PYZdq{}\PYZdq{}\PYZdq{}}
            \PY{k}{return} \PY{n}{cv2}\PY{o}{.}\PY{n}{GaussianBlur}\PY{p}{(}\PY{n}{img}\PY{p}{,} \PY{p}{(}\PY{n}{kernel\PYZus{}size}\PY{p}{,} \PY{n}{kernel\PYZus{}size}\PY{p}{)}\PY{p}{,} \PY{l+m+mi}{0}\PY{p}{)}
        
        \PY{k}{def} \PY{n+nf}{region\PYZus{}of\PYZus{}interest}\PY{p}{(}\PY{n}{img}\PY{p}{,} \PY{n}{vertices}\PY{p}{)}\PY{p}{:}
            \PY{l+s+sd}{\PYZdq{}\PYZdq{}\PYZdq{}}
        \PY{l+s+sd}{    Applies an image mask.}
        \PY{l+s+sd}{    }
        \PY{l+s+sd}{    Only keeps the region of the image defined by the polygon}
        \PY{l+s+sd}{    formed from `vertices`. The rest of the image is set to black.}
        \PY{l+s+sd}{    `vertices` should be a numpy array of integer points.}
        \PY{l+s+sd}{    \PYZdq{}\PYZdq{}\PYZdq{}}
            \PY{c+c1}{\PYZsh{}defining a blank mask to start with}
            \PY{n}{mask} \PY{o}{=} \PY{n}{np}\PY{o}{.}\PY{n}{zeros\PYZus{}like}\PY{p}{(}\PY{n}{img}\PY{p}{)}   
            
            \PY{c+c1}{\PYZsh{}defining a 3 channel or 1 channel color to fill the mask with depending on the input image}
            \PY{k}{if} \PY{n+nb}{len}\PY{p}{(}\PY{n}{img}\PY{o}{.}\PY{n}{shape}\PY{p}{)} \PY{o}{\PYZgt{}} \PY{l+m+mi}{2}\PY{p}{:}
                \PY{n}{channel\PYZus{}count} \PY{o}{=} \PY{n}{img}\PY{o}{.}\PY{n}{shape}\PY{p}{[}\PY{l+m+mi}{2}\PY{p}{]}  \PY{c+c1}{\PYZsh{} i.e. 3 or 4 depending on your image}
                \PY{n}{ignore\PYZus{}mask\PYZus{}color} \PY{o}{=} \PY{p}{(}\PY{l+m+mi}{255}\PY{p}{,}\PY{p}{)} \PY{o}{*} \PY{n}{channel\PYZus{}count}
            \PY{k}{else}\PY{p}{:}
                \PY{n}{ignore\PYZus{}mask\PYZus{}color} \PY{o}{=} \PY{l+m+mi}{255}
                
            \PY{c+c1}{\PYZsh{}filling pixels inside the polygon defined by \PYZdq{}vertices\PYZdq{} with the fill color    }
            \PY{n}{cv2}\PY{o}{.}\PY{n}{fillPoly}\PY{p}{(}\PY{n}{mask}\PY{p}{,} \PY{n}{vertices}\PY{p}{,} \PY{n}{ignore\PYZus{}mask\PYZus{}color}\PY{p}{)}
            
            \PY{c+c1}{\PYZsh{}returning the image only where mask pixels are nonzero}
            \PY{n}{masked\PYZus{}image} \PY{o}{=} \PY{n}{cv2}\PY{o}{.}\PY{n}{bitwise\PYZus{}and}\PY{p}{(}\PY{n}{img}\PY{p}{,} \PY{n}{mask}\PY{p}{)}
            \PY{k}{return} \PY{n}{masked\PYZus{}image}
        
        
        \PY{k}{def} \PY{n+nf}{draw\PYZus{}lines}\PY{p}{(}\PY{n}{img}\PY{p}{,} \PY{n}{lines}\PY{p}{,} \PY{n}{color}\PY{o}{=}\PY{p}{[}\PY{l+m+mi}{255}\PY{p}{,} \PY{l+m+mi}{0}\PY{p}{,} \PY{l+m+mi}{0}\PY{p}{]}\PY{p}{,} \PY{n}{thickness}\PY{o}{=}\PY{l+m+mi}{5}\PY{p}{)}\PY{p}{:}
            \PY{l+s+sd}{\PYZdq{}\PYZdq{}\PYZdq{}}
        \PY{l+s+sd}{    NOTE: this is the function you might want to use as a starting point once you want to }
        \PY{l+s+sd}{    average/extrapolate the line segments you detect to map out the full}
        \PY{l+s+sd}{    extent of the lane (going from the result shown in raw\PYZhy{}lines\PYZhy{}example.mp4}
        \PY{l+s+sd}{    to that shown in P1\PYZus{}example.mp4).  }
        \PY{l+s+sd}{    }
        \PY{l+s+sd}{    Think about things like separating line segments by their }
        \PY{l+s+sd}{    slope ((y2\PYZhy{}y1)/(x2\PYZhy{}x1)) to decide which segments are part of the left}
        \PY{l+s+sd}{    line vs. the right line.  Then, you can average the position of each of }
        \PY{l+s+sd}{    the lines and extrapolate to the top and bottom of the lane.}
        \PY{l+s+sd}{    }
        \PY{l+s+sd}{    This function draws `lines` with `color` and `thickness`.    }
        \PY{l+s+sd}{    Lines are drawn on the image inplace (mutates the image).}
        \PY{l+s+sd}{    If you want to make the lines semi\PYZhy{}transparent, think about combining}
        \PY{l+s+sd}{    this function with the weighted\PYZus{}img() function below}
        \PY{l+s+sd}{    \PYZdq{}\PYZdq{}\PYZdq{}}
            \PY{k}{for} \PY{n}{line} \PY{o+ow}{in} \PY{n}{lines}\PY{p}{:}
                \PY{k}{for} \PY{n}{x1}\PY{p}{,}\PY{n}{y1}\PY{p}{,}\PY{n}{x2}\PY{p}{,}\PY{n}{y2} \PY{o+ow}{in} \PY{n}{line}\PY{p}{:}
                    \PY{n}{cv2}\PY{o}{.}\PY{n}{line}\PY{p}{(}\PY{n}{img}\PY{p}{,} \PY{p}{(}\PY{n}{x1}\PY{p}{,} \PY{n}{y1}\PY{p}{)}\PY{p}{,} \PY{p}{(}\PY{n}{x2}\PY{p}{,} \PY{n}{y2}\PY{p}{)}\PY{p}{,} \PY{n}{color}\PY{p}{,} \PY{n}{thickness}\PY{p}{)}
        
        \PY{k}{def} \PY{n+nf}{hough\PYZus{}lines}\PY{p}{(}\PY{n}{img}\PY{p}{,} \PY{n}{rho}\PY{p}{,} \PY{n}{theta}\PY{p}{,} \PY{n}{threshold}\PY{p}{,} \PY{n}{min\PYZus{}line\PYZus{}len}\PY{p}{,} \PY{n}{max\PYZus{}line\PYZus{}gap}\PY{p}{)}\PY{p}{:}
            \PY{l+s+sd}{\PYZdq{}\PYZdq{}\PYZdq{}}
        \PY{l+s+sd}{    `img` should be the output of a Canny transform.}
        \PY{l+s+sd}{        }
        \PY{l+s+sd}{    Returns an image with hough lines drawn.}
        \PY{l+s+sd}{    \PYZdq{}\PYZdq{}\PYZdq{}}
            \PY{n}{lines} \PY{o}{=} \PY{n}{cv2}\PY{o}{.}\PY{n}{HoughLinesP}\PY{p}{(}\PY{n}{img}\PY{p}{,} \PY{n}{rho}\PY{p}{,} \PY{n}{theta}\PY{p}{,} \PY{n}{threshold}\PY{p}{,} \PY{n}{np}\PY{o}{.}\PY{n}{array}\PY{p}{(}\PY{p}{[}\PY{p}{]}\PY{p}{)}\PY{p}{,} \PY{n}{minLineLength}\PY{o}{=}\PY{n}{min\PYZus{}line\PYZus{}len}\PY{p}{,} \PY{n}{maxLineGap}\PY{o}{=}\PY{n}{max\PYZus{}line\PYZus{}gap}\PY{p}{)}
            \PY{n}{extended\PYZus{}lines} \PY{o}{=} \PY{n}{extend\PYZus{}line}\PY{p}{(}\PY{n}{lines}\PY{p}{)}
            \PY{n}{line\PYZus{}img} \PY{o}{=} \PY{n}{np}\PY{o}{.}\PY{n}{zeros}\PY{p}{(}\PY{p}{(}\PY{n}{img}\PY{o}{.}\PY{n}{shape}\PY{p}{[}\PY{l+m+mi}{0}\PY{p}{]}\PY{p}{,} \PY{n}{img}\PY{o}{.}\PY{n}{shape}\PY{p}{[}\PY{l+m+mi}{1}\PY{p}{]}\PY{p}{,} \PY{l+m+mi}{3}\PY{p}{)}\PY{p}{,} \PY{n}{dtype}\PY{o}{=}\PY{n}{np}\PY{o}{.}\PY{n}{uint8}\PY{p}{)}
            \PY{n}{draw\PYZus{}lines}\PY{p}{(}\PY{n}{line\PYZus{}img}\PY{p}{,} \PY{n}{extended\PYZus{}lines}\PY{p}{)}
            \PY{k}{return} \PY{n}{line\PYZus{}img}
        
        \PY{c+c1}{\PYZsh{} Python 3 has support for cool math symbols.}
        
        \PY{k}{def} \PY{n+nf}{weighted\PYZus{}img}\PY{p}{(}\PY{n}{img}\PY{p}{,} \PY{n}{initial\PYZus{}img}\PY{p}{,} \PY{n}{α}\PY{o}{=}\PY{l+m+mf}{0.8}\PY{p}{,} \PY{n}{β}\PY{o}{=}\PY{l+m+mf}{1.}\PY{p}{,} \PY{n}{γ}\PY{o}{=}\PY{l+m+mf}{0.}\PY{p}{)}\PY{p}{:}
            \PY{l+s+sd}{\PYZdq{}\PYZdq{}\PYZdq{}}
        \PY{l+s+sd}{    `img` is the output of the hough\PYZus{}lines(), An image with lines drawn on it.}
        \PY{l+s+sd}{    Should be a blank image (all black) with lines drawn on it.}
        \PY{l+s+sd}{    }
        \PY{l+s+sd}{    `initial\PYZus{}img` should be the image before any processing.}
        \PY{l+s+sd}{    }
        \PY{l+s+sd}{    The result image is computed as follows:}
        \PY{l+s+sd}{    }
        \PY{l+s+sd}{    initial\PYZus{}img * α + img * β + γ}
        \PY{l+s+sd}{    NOTE: initial\PYZus{}img and img must be the same shape!}
        \PY{l+s+sd}{    \PYZdq{}\PYZdq{}\PYZdq{}}
            \PY{k}{return} \PY{n}{cv2}\PY{o}{.}\PY{n}{addWeighted}\PY{p}{(}\PY{n}{initial\PYZus{}img}\PY{p}{,} \PY{n}{α}\PY{p}{,} \PY{n}{img}\PY{p}{,} \PY{n}{β}\PY{p}{,} \PY{n}{γ}\PY{p}{)}
\end{Verbatim}


    \hypertarget{test-images}{%
\subsection{Test Images}\label{test-images}}

Build your pipeline to work on the images in the directory
``test\_images''\\
\textbf{You should make sure your pipeline works well on these images
before you try the videos.}

    \begin{Verbatim}[commandchars=\\\{\}]
{\color{incolor}In [{\color{incolor}7}]:} \PY{k+kn}{import} \PY{n+nn}{os}
        \PY{n}{lanes} \PY{o}{=} \PY{n}{os}\PY{o}{.}\PY{n}{listdir}\PY{p}{(}\PY{l+s+s2}{\PYZdq{}}\PY{l+s+s2}{test\PYZus{}images/}\PY{l+s+s2}{\PYZdq{}}\PY{p}{)}
\end{Verbatim}


    \hypertarget{build-a-lane-finding-pipeline}{%
\subsection{Build a Lane Finding
Pipeline}\label{build-a-lane-finding-pipeline}}

    Build the pipeline and run your solution on all test\_images. Make
copies into the \texttt{test\_images\_output} directory, and you can use
the images in your writeup report.

Try tuning the various parameters, especially the low and high Canny
thresholds as well as the Hough lines parameters.

    \begin{Verbatim}[commandchars=\\\{\}]
{\color{incolor}In [{\color{incolor}8}]:} \PY{c+c1}{\PYZsh{} TODO: Build your pipeline that will draw lane lines on the test\PYZus{}images}
        \PY{c+c1}{\PYZsh{} then save them to the test\PYZus{}images\PYZus{}output directory.}
        
        \PY{k}{def} \PY{n+nf}{draw\PYZus{}lane}\PY{p}{(}\PY{n}{image}\PY{p}{)}\PY{p}{:}
            \PY{c+c1}{\PYZsh{} Read image}
            \PY{n}{imshape} \PY{o}{=} \PY{n}{image}\PY{o}{.}\PY{n}{shape}
        
            \PY{c+c1}{\PYZsh{} Convert image to grayscale}
            \PY{n}{grayed\PYZus{}image} \PY{o}{=} \PY{n}{grayscale}\PY{p}{(}\PY{n}{image}\PY{p}{)}
        
            \PY{c+c1}{\PYZsh{} Define a kernel size and apply Gaussian smoothing}
            \PY{n}{kernel\PYZus{}size} \PY{o}{=} \PY{l+m+mi}{5}
            \PY{n}{blur\PYZus{}gray} \PY{o}{=} \PY{n}{gaussian\PYZus{}blur}\PY{p}{(}\PY{n}{img}\PY{o}{=}\PY{n}{grayed\PYZus{}image}\PY{p}{,} \PY{n}{kernel\PYZus{}size}\PY{o}{=}\PY{n}{kernel\PYZus{}size}\PY{p}{)}
        
        
            \PY{c+c1}{\PYZsh{} Define parameters for Canny and apply}
            \PY{n}{low\PYZus{}threshold} \PY{o}{=} \PY{l+m+mi}{50}
            \PY{n}{high\PYZus{}threshold} \PY{o}{=} \PY{l+m+mi}{150}
            \PY{n}{edges} \PY{o}{=} \PY{n}{canny}\PY{p}{(}\PY{n}{img}\PY{o}{=}\PY{n}{blur\PYZus{}gray}\PY{p}{,} \PY{n}{low\PYZus{}threshold}\PY{o}{=}\PY{n}{low\PYZus{}threshold}\PY{p}{,} \PY{n}{high\PYZus{}threshold}\PY{o}{=}\PY{n}{high\PYZus{}threshold}\PY{p}{)} 
        
        
            \PY{c+c1}{\PYZsh{} mask the image}
            \PY{n}{vertices} \PY{o}{=} \PY{n}{np}\PY{o}{.}\PY{n}{array}\PY{p}{(}\PY{p}{[}\PY{p}{[}\PY{p}{(}\PY{l+m+mi}{150}\PY{p}{,}\PY{n}{imshape}\PY{p}{[}\PY{l+m+mi}{0}\PY{p}{]}\PY{p}{)}\PY{p}{,}\PY{p}{(}\PY{l+m+mi}{450}\PY{p}{,} \PY{l+m+mi}{300}\PY{p}{)}\PY{p}{,} \PY{p}{(}\PY{l+m+mi}{490}\PY{p}{,} \PY{l+m+mi}{300}\PY{p}{)}\PY{p}{,} \PY{p}{(}\PY{n}{imshape}\PY{p}{[}\PY{l+m+mi}{1}\PY{p}{]}\PY{p}{,}\PY{n}{imshape}\PY{p}{[}\PY{l+m+mi}{0}\PY{p}{]}\PY{p}{)}\PY{p}{]}\PY{p}{]}\PY{p}{,} \PY{n}{dtype}\PY{o}{=}\PY{n}{np}\PY{o}{.}\PY{n}{int32}\PY{p}{)}
            \PY{n}{masked\PYZus{}edges} \PY{o}{=} \PY{n}{region\PYZus{}of\PYZus{}interest}\PY{p}{(}\PY{n}{img}\PY{o}{=}\PY{n}{edges}\PY{p}{,} \PY{n}{vertices}\PY{o}{=}\PY{n}{vertices}\PY{p}{)}
        
        
            \PY{c+c1}{\PYZsh{} Define the Hough transform parameters}
            \PY{c+c1}{\PYZsh{} Make a blank image of the same size as the original image to draw on}
            \PY{n}{rho} \PY{o}{=} \PY{l+m+mi}{2}
            \PY{n}{theta} \PY{o}{=} \PY{n}{np}\PY{o}{.}\PY{n}{pi}\PY{o}{/}\PY{l+m+mi}{180} 
            \PY{n}{threshold} \PY{o}{=} \PY{l+m+mi}{20}
            \PY{n}{min\PYZus{}line\PYZus{}length} \PY{o}{=} \PY{l+m+mi}{120}
            \PY{n}{max\PYZus{}line\PYZus{}gap} \PY{o}{=} \PY{l+m+mi}{60}
        
            \PY{n}{lines} \PY{o}{=} \PY{n}{hough\PYZus{}lines}\PY{p}{(}\PY{n}{img}\PY{o}{=}\PY{n}{masked\PYZus{}edges}\PY{p}{,} \PY{n}{rho}\PY{o}{=}\PY{n}{rho}\PY{p}{,} \PY{n}{theta}\PY{o}{=}\PY{n}{theta}\PY{p}{,} \PY{n}{threshold}\PY{o}{=}\PY{n}{threshold}\PY{p}{,} \PY{n}{min\PYZus{}line\PYZus{}len}\PY{o}{=}\PY{n}{min\PYZus{}line\PYZus{}length}\PY{p}{,}
                                \PY{n}{max\PYZus{}line\PYZus{}gap}\PY{o}{=}\PY{n}{max\PYZus{}line\PYZus{}gap}\PY{p}{)}
        
            \PY{c+c1}{\PYZsh{} Create a \PYZdq{}color\PYZdq{} binary image to combine with line image}
            \PY{n}{color\PYZus{}edges} \PY{o}{=} \PY{n}{np}\PY{o}{.}\PY{n}{dstack}\PY{p}{(}\PY{p}{(}\PY{n}{edges}\PY{p}{,} \PY{n}{edges}\PY{p}{,} \PY{n}{edges}\PY{p}{)}\PY{p}{)} 
        
            \PY{c+c1}{\PYZsh{} Draw the lines on the edge image}
            \PY{n}{lines\PYZus{}edges} \PY{o}{=} \PY{n}{weighted\PYZus{}img}\PY{p}{(}\PY{n}{lines}\PY{p}{,} \PY{n}{image}\PY{p}{)}
            \PY{k}{return} \PY{n}{lines\PYZus{}edges}
\end{Verbatim}


    \begin{Verbatim}[commandchars=\\\{\}]
{\color{incolor}In [{\color{incolor}9}]:} \PY{k}{for} \PY{n}{lane} \PY{o+ow}{in} \PY{n}{lanes}\PY{p}{:}
            \PY{n}{image} \PY{o}{=} \PY{n}{mpimg}\PY{o}{.}\PY{n}{imread}\PY{p}{(}\PY{l+s+s2}{\PYZdq{}}\PY{l+s+s2}{test\PYZus{}images/}\PY{l+s+si}{\PYZob{}\PYZcb{}}\PY{l+s+s2}{\PYZdq{}}\PY{o}{.}\PY{n}{format}\PY{p}{(}\PY{n}{lane}\PY{p}{)}\PY{p}{)}
            \PY{n}{lines\PYZus{}edges} \PY{o}{=} \PY{n}{draw\PYZus{}lane}\PY{p}{(}\PY{n}{image}\PY{p}{)}
            \PY{n}{plt}\PY{o}{.}\PY{n}{title}\PY{p}{(}\PY{n}{lane}\PY{p}{)}
            \PY{n}{plt}\PY{o}{.}\PY{n}{imshow}\PY{p}{(}\PY{n}{lines\PYZus{}edges}\PY{p}{)}
            \PY{n}{cv2}\PY{o}{.}\PY{n}{imwrite}\PY{p}{(}\PY{l+s+s2}{\PYZdq{}}\PY{l+s+s2}{test\PYZus{}images\PYZus{}output/}\PY{l+s+si}{\PYZob{}\PYZcb{}}\PY{l+s+s2}{\PYZdq{}}\PY{o}{.}\PY{n}{format}\PY{p}{(}\PY{n}{lane}\PY{p}{)}\PY{p}{,} \PY{n}{cv2}\PY{o}{.}\PY{n}{cvtColor}\PY{p}{(}\PY{n}{lines\PYZus{}edges}\PY{p}{,} \PY{n}{cv2}\PY{o}{.}\PY{n}{COLOR\PYZus{}RGB2BGR}\PY{p}{)}\PY{p}{)}
            \PY{n}{plt}\PY{o}{.}\PY{n}{show}\PY{p}{(}\PY{p}{)}
\end{Verbatim}


    \begin{Verbatim}[commandchars=\\\{\}]
/Users/vkv/anaconda3/envs/carnd-term1/lib/python3.5/site-packages/ipykernel\_launcher.py:5: RuntimeWarning: invalid value encountered in long\_scalars
  """

    \end{Verbatim}

    \begin{center}
    \adjustimage{max size={0.9\linewidth}{0.9\paperheight}}{output_20_1.png}
    \end{center}
    { \hspace*{\fill} \\}
    
    \begin{center}
    \adjustimage{max size={0.9\linewidth}{0.9\paperheight}}{output_20_2.png}
    \end{center}
    { \hspace*{\fill} \\}
    
    \begin{center}
    \adjustimage{max size={0.9\linewidth}{0.9\paperheight}}{output_20_3.png}
    \end{center}
    { \hspace*{\fill} \\}
    
    \begin{center}
    \adjustimage{max size={0.9\linewidth}{0.9\paperheight}}{output_20_4.png}
    \end{center}
    { \hspace*{\fill} \\}
    
    \begin{center}
    \adjustimage{max size={0.9\linewidth}{0.9\paperheight}}{output_20_5.png}
    \end{center}
    { \hspace*{\fill} \\}
    
    \begin{center}
    \adjustimage{max size={0.9\linewidth}{0.9\paperheight}}{output_20_6.png}
    \end{center}
    { \hspace*{\fill} \\}
    
    \hypertarget{test-on-videos}{%
\subsection{Test on Videos}\label{test-on-videos}}

You know what's cooler than drawing lanes over images? Drawing lanes
over video!

We can test our solution on two provided videos:

\texttt{solidWhiteRight.mp4}

\texttt{solidYellowLeft.mp4}

\textbf{Note: if you get an import error when you run the next cell, try
changing your kernel (select the Kernel menu above --\textgreater{}
Change Kernel). Still have problems? Try relaunching Jupyter Notebook
from the terminal prompt. Also, consult the forums for more
troubleshooting tips.}

\textbf{If you get an error that looks like this:}

\begin{verbatim}
NeedDownloadError: Need ffmpeg exe. 
You can download it by calling: 
imageio.plugins.ffmpeg.download()
\end{verbatim}

\textbf{Follow the instructions in the error message and check out
\href{https://discussions.udacity.com/t/project-error-of-test-on-videos/274082}{this
forum post} for more troubleshooting tips across operating systems.}

    \begin{Verbatim}[commandchars=\\\{\}]
{\color{incolor}In [{\color{incolor}10}]:} \PY{c+c1}{\PYZsh{} Import everything needed to edit/save/watch video clips}
         \PY{k+kn}{from} \PY{n+nn}{moviepy}\PY{n+nn}{.}\PY{n+nn}{editor} \PY{k}{import} \PY{n}{VideoFileClip}
         \PY{k+kn}{from} \PY{n+nn}{IPython}\PY{n+nn}{.}\PY{n+nn}{display} \PY{k}{import} \PY{n}{HTML}
\end{Verbatim}


    \begin{Verbatim}[commandchars=\\\{\}]
{\color{incolor}In [{\color{incolor}11}]:} \PY{k}{def} \PY{n+nf}{process\PYZus{}image}\PY{p}{(}\PY{n}{image}\PY{p}{)}\PY{p}{:}
             \PY{c+c1}{\PYZsh{} NOTE: The output you return should be a color image (3 channel) for processing video below}
             \PY{c+c1}{\PYZsh{} TODO: put your pipeline here,}
             \PY{c+c1}{\PYZsh{} you should return the final output (image where lines are drawn on lanes)}
             \PY{k}{return} \PY{n}{draw\PYZus{}lane}\PY{p}{(}\PY{n}{image}\PY{p}{)}
\end{Verbatim}


    Let's try the one with the solid white lane on the right first \ldots{}

    \begin{Verbatim}[commandchars=\\\{\}]
{\color{incolor}In [{\color{incolor}12}]:} \PY{n}{white\PYZus{}output} \PY{o}{=} \PY{l+s+s1}{\PYZsq{}}\PY{l+s+s1}{test\PYZus{}videos\PYZus{}output/solidWhiteRight.mp4}\PY{l+s+s1}{\PYZsq{}}
         \PY{c+c1}{\PYZsh{}\PYZsh{} To speed up the testing process you may want to try your pipeline on a shorter subclip of the video}
         \PY{c+c1}{\PYZsh{}\PYZsh{} To do so add .subclip(start\PYZus{}second,end\PYZus{}second) to the end of the line below}
         \PY{c+c1}{\PYZsh{}\PYZsh{} Where start\PYZus{}second and end\PYZus{}second are integer values representing the start and end of the subclip}
         \PY{c+c1}{\PYZsh{}\PYZsh{} You may also uncomment the following line for a subclip of the first 5 seconds}
         \PY{c+c1}{\PYZsh{}\PYZsh{}clip1 = VideoFileClip(\PYZdq{}test\PYZus{}videos/solidWhiteRight.mp4\PYZdq{}).subclip(0,5)}
         \PY{n}{clip1} \PY{o}{=} \PY{n}{VideoFileClip}\PY{p}{(}\PY{l+s+s2}{\PYZdq{}}\PY{l+s+s2}{test\PYZus{}videos/solidWhiteRight.mp4}\PY{l+s+s2}{\PYZdq{}}\PY{p}{)}
         \PY{n}{white\PYZus{}clip} \PY{o}{=} \PY{n}{clip1}\PY{o}{.}\PY{n}{fl\PYZus{}image}\PY{p}{(}\PY{n}{process\PYZus{}image}\PY{p}{)} \PY{c+c1}{\PYZsh{}NOTE: this function expects color images!!}
         \PY{o}{\PYZpc{}}\PY{k}{time} white\PYZus{}clip.write\PYZus{}videofile(white\PYZus{}output, audio=False)
\end{Verbatim}


    \begin{Verbatim}[commandchars=\\\{\}]
[MoviePy] >>>> Building video test\_videos\_output/solidWhiteRight.mp4
[MoviePy] Writing video test\_videos\_output/solidWhiteRight.mp4

    \end{Verbatim}

    \begin{Verbatim}[commandchars=\\\{\}]
100\%|█████████▉| 221/222 [00:05<00:00, 37.73it/s]

    \end{Verbatim}

    \begin{Verbatim}[commandchars=\\\{\}]
[MoviePy] Done.
[MoviePy] >>>> Video ready: test\_videos\_output/solidWhiteRight.mp4 

CPU times: user 6.86 s, sys: 1.12 s, total: 7.98 s
Wall time: 6.12 s

    \end{Verbatim}

    Play the video inline, or if you prefer find the video in your
filesystem (should be in the same directory) and play it in your video
player of choice.

    \begin{Verbatim}[commandchars=\\\{\}]
{\color{incolor}In [{\color{incolor}13}]:} \PY{n}{HTML}\PY{p}{(}\PY{l+s+s2}{\PYZdq{}\PYZdq{}\PYZdq{}}
         \PY{l+s+s2}{\PYZlt{}video width=}\PY{l+s+s2}{\PYZdq{}}\PY{l+s+s2}{960}\PY{l+s+s2}{\PYZdq{}}\PY{l+s+s2}{ height=}\PY{l+s+s2}{\PYZdq{}}\PY{l+s+s2}{540}\PY{l+s+s2}{\PYZdq{}}\PY{l+s+s2}{ controls\PYZgt{}}
         \PY{l+s+s2}{  \PYZlt{}source src=}\PY{l+s+s2}{\PYZdq{}}\PY{l+s+si}{\PYZob{}0\PYZcb{}}\PY{l+s+s2}{\PYZdq{}}\PY{l+s+s2}{\PYZgt{}}
         \PY{l+s+s2}{\PYZlt{}/video\PYZgt{}}
         \PY{l+s+s2}{\PYZdq{}\PYZdq{}\PYZdq{}}\PY{o}{.}\PY{n}{format}\PY{p}{(}\PY{n}{white\PYZus{}output}\PY{p}{)}\PY{p}{)}
\end{Verbatim}


\begin{Verbatim}[commandchars=\\\{\}]
{\color{outcolor}Out[{\color{outcolor}13}]:} <IPython.core.display.HTML object>
\end{Verbatim}
            
    \hypertarget{improve-the-draw_lines-function}{%
\subsection{Improve the draw\_lines()
function}\label{improve-the-draw_lines-function}}

\textbf{At this point, if you were successful with making the pipeline
and tuning parameters, you probably have the Hough line segments drawn
onto the road, but what about identifying the full extent of the lane
and marking it clearly as in the example video (P1\_example.mp4)? Think
about defining a line to run the full length of the visible lane based
on the line segments you identified with the Hough Transform. As
mentioned previously, try to average and/or extrapolate the line
segments you've detected to map out the full extent of the lane lines.
You can see an example of the result you're going for in the video
``P1\_example.mp4''.}

\textbf{Go back and modify your draw\_lines function accordingly and try
re-running your pipeline. The new output should draw a single, solid
line over the left lane line and a single, solid line over the right
lane line. The lines should start from the bottom of the image and
extend out to the top of the region of interest.}

    Now for the one with the solid yellow lane on the left. This one's more
tricky!

    \begin{Verbatim}[commandchars=\\\{\}]
{\color{incolor}In [{\color{incolor}14}]:} \PY{n}{yellow\PYZus{}output} \PY{o}{=} \PY{l+s+s1}{\PYZsq{}}\PY{l+s+s1}{test\PYZus{}videos\PYZus{}output/solidYellowLeft.mp4}\PY{l+s+s1}{\PYZsq{}}
         \PY{c+c1}{\PYZsh{}\PYZsh{} To speed up the testing process you may want to try your pipeline on a shorter subclip of the video}
         \PY{c+c1}{\PYZsh{}\PYZsh{} To do so add .subclip(start\PYZus{}second,end\PYZus{}second) to the end of the line below}
         \PY{c+c1}{\PYZsh{}\PYZsh{} Where start\PYZus{}second and end\PYZus{}second are integer values representing the start and end of the subclip}
         \PY{c+c1}{\PYZsh{}\PYZsh{} You may also uncomment the following line for a subclip of the first 5 seconds}
         \PY{c+c1}{\PYZsh{}\PYZsh{}clip2 = VideoFileClip(\PYZsq{}test\PYZus{}videos/solidYellowLeft.mp4\PYZsq{}).subclip(0,5)}
         \PY{n}{clip2} \PY{o}{=} \PY{n}{VideoFileClip}\PY{p}{(}\PY{l+s+s1}{\PYZsq{}}\PY{l+s+s1}{test\PYZus{}videos/solidYellowLeft.mp4}\PY{l+s+s1}{\PYZsq{}}\PY{p}{)}
         \PY{n}{yellow\PYZus{}clip} \PY{o}{=} \PY{n}{clip2}\PY{o}{.}\PY{n}{fl\PYZus{}image}\PY{p}{(}\PY{n}{process\PYZus{}image}\PY{p}{)}
         \PY{o}{\PYZpc{}}\PY{k}{time} yellow\PYZus{}clip.write\PYZus{}videofile(yellow\PYZus{}output, audio=False)
\end{Verbatim}


    \begin{Verbatim}[commandchars=\\\{\}]
[MoviePy] >>>> Building video test\_videos\_output/solidYellowLeft.mp4
[MoviePy] Writing video test\_videos\_output/solidYellowLeft.mp4

    \end{Verbatim}

    \begin{Verbatim}[commandchars=\\\{\}]
100\%|█████████▉| 681/682 [00:17<00:00, 39.02it/s]

    \end{Verbatim}

    \begin{Verbatim}[commandchars=\\\{\}]
[MoviePy] Done.
[MoviePy] >>>> Video ready: test\_videos\_output/solidYellowLeft.mp4 

CPU times: user 20.5 s, sys: 3.43 s, total: 23.9 s
Wall time: 17.7 s

    \end{Verbatim}

    \begin{Verbatim}[commandchars=\\\{\}]
{\color{incolor}In [{\color{incolor}15}]:} \PY{n}{HTML}\PY{p}{(}\PY{l+s+s2}{\PYZdq{}\PYZdq{}\PYZdq{}}
         \PY{l+s+s2}{\PYZlt{}video width=}\PY{l+s+s2}{\PYZdq{}}\PY{l+s+s2}{960}\PY{l+s+s2}{\PYZdq{}}\PY{l+s+s2}{ height=}\PY{l+s+s2}{\PYZdq{}}\PY{l+s+s2}{540}\PY{l+s+s2}{\PYZdq{}}\PY{l+s+s2}{ controls\PYZgt{}}
         \PY{l+s+s2}{  \PYZlt{}source src=}\PY{l+s+s2}{\PYZdq{}}\PY{l+s+si}{\PYZob{}0\PYZcb{}}\PY{l+s+s2}{\PYZdq{}}\PY{l+s+s2}{\PYZgt{}}
         \PY{l+s+s2}{\PYZlt{}/video\PYZgt{}}
         \PY{l+s+s2}{\PYZdq{}\PYZdq{}\PYZdq{}}\PY{o}{.}\PY{n}{format}\PY{p}{(}\PY{n}{yellow\PYZus{}output}\PY{p}{)}\PY{p}{)}
\end{Verbatim}


\begin{Verbatim}[commandchars=\\\{\}]
{\color{outcolor}Out[{\color{outcolor}15}]:} <IPython.core.display.HTML object>
\end{Verbatim}
            
    \hypertarget{writeup-and-submission}{%
\subsection{Writeup and Submission}\label{writeup-and-submission}}

If you're satisfied with your video outputs, it's time to make the
report writeup in a pdf or markdown file. Once you have this Ipython
notebook ready along with the writeup, it's time to submit for review!
Here is a
\href{https://github.com/udacity/CarND-LaneLines-P1/blob/master/writeup_template.md}{link}
to the writeup template file.

    \hypertarget{optional-challenge}{%
\subsection{Optional Challenge}\label{optional-challenge}}

Try your lane finding pipeline on the video below. Does it still work?
Can you figure out a way to make it more robust? If you're up for the
challenge, modify your pipeline so it works with this video and submit
it along with the rest of your project!

    \begin{Verbatim}[commandchars=\\\{\}]
{\color{incolor}In [{\color{incolor} }]:} \PY{n}{challenge\PYZus{}output} \PY{o}{=} \PY{l+s+s1}{\PYZsq{}}\PY{l+s+s1}{test\PYZus{}videos\PYZus{}output/challenge.mp4}\PY{l+s+s1}{\PYZsq{}}
        \PY{c+c1}{\PYZsh{}\PYZsh{} To speed up the testing process you may want to try your pipeline on a shorter subclip of the video}
        \PY{c+c1}{\PYZsh{}\PYZsh{} To do so add .subclip(start\PYZus{}second,end\PYZus{}second) to the end of the line below}
        \PY{c+c1}{\PYZsh{}\PYZsh{} Where start\PYZus{}second and end\PYZus{}second are integer values representing the start and end of the subclip}
        \PY{c+c1}{\PYZsh{}\PYZsh{} You may also uncomment the following line for a subclip of the first 5 seconds}
        \PY{c+c1}{\PYZsh{}\PYZsh{}clip3 = VideoFileClip(\PYZsq{}test\PYZus{}videos/challenge.mp4\PYZsq{}).subclip(0,5)}
        \PY{n}{clip3} \PY{o}{=} \PY{n}{VideoFileClip}\PY{p}{(}\PY{l+s+s1}{\PYZsq{}}\PY{l+s+s1}{test\PYZus{}videos/challenge.mp4}\PY{l+s+s1}{\PYZsq{}}\PY{p}{)}
        \PY{n}{challenge\PYZus{}clip} \PY{o}{=} \PY{n}{clip3}\PY{o}{.}\PY{n}{fl\PYZus{}image}\PY{p}{(}\PY{n}{process\PYZus{}image}\PY{p}{)}
        \PY{o}{\PYZpc{}}\PY{k}{time} challenge\PYZus{}clip.write\PYZus{}videofile(challenge\PYZus{}output, audio=False)
\end{Verbatim}


    \begin{Verbatim}[commandchars=\\\{\}]
{\color{incolor}In [{\color{incolor}17}]:} \PY{n}{HTML}\PY{p}{(}\PY{l+s+s2}{\PYZdq{}\PYZdq{}\PYZdq{}}
         \PY{l+s+s2}{\PYZlt{}video width=}\PY{l+s+s2}{\PYZdq{}}\PY{l+s+s2}{960}\PY{l+s+s2}{\PYZdq{}}\PY{l+s+s2}{ height=}\PY{l+s+s2}{\PYZdq{}}\PY{l+s+s2}{540}\PY{l+s+s2}{\PYZdq{}}\PY{l+s+s2}{ controls\PYZgt{}}
         \PY{l+s+s2}{  \PYZlt{}source src=}\PY{l+s+s2}{\PYZdq{}}\PY{l+s+si}{\PYZob{}0\PYZcb{}}\PY{l+s+s2}{\PYZdq{}}\PY{l+s+s2}{\PYZgt{}}
         \PY{l+s+s2}{\PYZlt{}/video\PYZgt{}}
         \PY{l+s+s2}{\PYZdq{}\PYZdq{}\PYZdq{}}\PY{o}{.}\PY{n}{format}\PY{p}{(}\PY{n}{challenge\PYZus{}output}\PY{p}{)}\PY{p}{)}
\end{Verbatim}


\begin{Verbatim}[commandchars=\\\{\}]
{\color{outcolor}Out[{\color{outcolor}17}]:} <IPython.core.display.HTML object>
\end{Verbatim}
            
    \begin{Verbatim}[commandchars=\\\{\}]
 16\%|█▌        | 40/251 [00:20<00:14, 14.85it/s]
    \end{Verbatim}


    % Add a bibliography block to the postdoc
    
    
    
    \end{document}
